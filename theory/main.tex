\documentclass[paper=a4,11pt,headsepline]{scrartcl}

\usepackage[margin=2cm]{geometry}
\usepackage[fleqn]{amsmath}
\usepackage{amssymb}
\usepackage{bm}
\usepackage{minted}
\usepackage{booktabs}
\usepackage{url}
\usepackage[
    colorlinks=true,
    linkcolor=black,
    citecolor=black,
    urlcolor=black,
    hyperfootnotes=true,
]{hyperref}
\usepackage{cancel}
\usepackage{xspace}
\usepackage{cleveref}
\usepackage{environ}
\usepackage[
    backend=biber,
    maxbibnames=2,
    giveninits=true,
    url=true,
    isbn=false,
    sorting=none,
    date=year,
]{biblatex}

\addbibresource{lit.bib}

\newcommand{\ve}[1]{\ensuremath{\bm{\mathit{#1}}}}
\newcommand{\ma}[1]{\ensuremath{\bm{\mathbf{#1}}}}
\newcommand{\dd}{\text{d}}
\newcommand{\ra}{\ensuremath{\rightarrow}\xspace}
\newcommand{\dra}{\ensuremath{\Rightarrow}\xspace}
\newcommand{\la}{\ensuremath{\leftarrow}\xspace}
\newcommand{\pd}[2]{\dfrac{\partial #1}{\partial #2}}
\newcommand{\td}[2]{\dfrac{\dd #1}{\dd #2}}
\newcommand{\pdi}[2]{\partial #1/\partial #2}
\newcommand{\tdi}[2]{\dd #1/\dd #2}
\newcommand{\red}[1]{{\color{red}{#1}}}
\newcommand{\green}[1]{{\color{green}{#1}}}
\newcommand{\blue}[1]{{\color{blue}{#1}}}
\newcommand{\ts}[1]{\textsuperscript{#1}}
\newcommand{\soft}[1]{\textsl{#1}\xspace}
\newcommand{\numpy}{\soft{numpy}}
\newcommand{\pytorch}{\soft{PyTorch}}
\newcommand{\jax}{\soft{JAX}}
\newcommand{\autograd}{\soft{autograd}}
\newcommand{\scipy}{\soft{scipy}}
\newcommand{\numdifftools}{\soft{numdifftools}}
\newcommand{\tf}{\soft{TensorFlow}}
\newcommand{\sympy}{\soft{sympy}}
\newcommand{\co}[1]{\texttt{#1}}
\newcommand{\norm}[1]{\left\lVert#1\right\rVert}
\newcommand{\eqdot}{\:.}


\newcommand{\ipmpy}[1]{\inputminted[xleftmargin=0.9cm]{python}{#1}}

% https://tex.stackexchange.com/a/5652
\NewEnviron{splitequation}{%
    \begin{equation}
        \begin{split}
            \BODY
        \end{split}
    \end{equation}
}



\begin{document}

\tableofcontents
\newpage

\section{Introduction}

This text is a short introduction to Automatic Differentiation (AD). The main
references are \cite{baydin_2018, johnson_2017, jax_autodiff_cookbook}. We
focus (for now) on basics linear computation graphs to introduce forward and
reverse mode, as well as differences between \jax and \pytorch (and a bit of
the older \autograd, the predecessor to \jax and one of the bases for
\pytorch's AD system \cite{paszke_2017}).

\section{Methods to evaluate derivatives}

This overview is inspired mostly by \cite{baydin_2018}.

\begin{itemize}
    \item Work out analytic derivatives by hand (or really Mathematica, Maple, Maxima, ...)
        and code them. Can be an advantage when the expression you want to
        differentiate never changes, e.g. derivatives of classical force
        fields in molecular simulations. Code of derivatives can be
        hand-optimized for speed.
    \item Call symbolic engines (e.g. \soft{sympy}) in your code to get
        analytic derivatives.
    \item Numerical derivatives by finite differences (FD). Use at least
        the central differences method. Good for checking AD or manually coded
        derivatives. Scales as $\mathcal O(n)$ for $\nabla f(\ve x), \ve
        x\in\mathbb R^n$.
    \item AD
        \begin{itemize}
            \item Directly provides numerical values for derivatives at some point
                $\ve x$ \emph{with machine precision}, \emph{not} symbolic expressions of
                derivatives
            \item Uses operator overloading (or other methods) to carry out
                derivatives (\jax, \pytorch: define JVPs/VJPs, see below)
            \item Can be applied to almost any code using arbitrary control flows
            \item In reverse mode: uses code tracing ("taping") by one forward eval
                to construct computation graph
            \item No "code swell": complex analytic expressions that grow which each
                application of $\partial/\partial x_i$ in symbolic engines and that
                need an additional simplification step
        \end{itemize}
\end{itemize}

\section{Forward and reverse AD}

\subsection{Setting the stage}

First consider a scalar function of scalar inputs $f:\mathbb R\ra\mathbb R$
\begin{equation}
    f(x) = c(b(a(x)))
\end{equation}
which implies the computation graph $x\ra a\ra b\ra c = f$. Forward = $x \ra c$ (inputs
to outputs). Reverse = $x \la c$ (outputs to inputs).
The derivative using the chain rule is given by
\begin{splitequation}
    \td{f}{x}
        &= \td{c}{b}\,\td{b}{a}\,\td{a}{x}\\
        &= \td{c}{b}\,\left(\td{b}{a}\,\td{a}{x}\right)&&\quad\text{forward}\\
        &= \left(\td{c}{b}\,\td{b}{a}\right)\,\td{a}{x}&&\quad\text{reverse}
\end{splitequation}

Now we consider the general case of a multivariate vector-valued function $\ve
f: \mathbb R^n \ra \mathbb R^m$. The Jacobian is the matrix of all first
partial derivatives and is defined as
%
\begin{equation}
    \ve f: \mathbb R^n \ra \mathbb R^m,\quad \ma J
    = \pd{\ve f}{\ve x} =
    \begin{bmatrix}
        \pd{f_1}{x_1} & \cdots & \pd{f_1}{x_n}  \\
        \vdots        & \ddots & \vdots         \\
        \pd{f_m}{x_1} & \cdots & \pd{f_m}{x_n}
    \end{bmatrix}
    \in\mathbb R^{m\times n}
    \eqdot
\end{equation}
One edge case is
\begin{equation}
    \ve f: \mathbb R^1 \ra \mathbb R^m\quad \ma J
    = \pd{\ve f}{x} =
    \begin{bmatrix}
        \pd{f_1}{x_1}  \\
        \vdots         \\
        \pd{f_m}{x_1}
    \end{bmatrix}
    = \ma J[:,1]\in\mathbb R^{m\times 1}
\end{equation}
where we need only the first column $\ma J[:,1]$. In the other edge case
\begin{equation}
    f: \mathbb R^n \ra \mathbb R^1\quad \ma J
    = \pd{f}{\ve x} =
    \begin{bmatrix}
        \pd{f_1}{x_1} & \cdots & \pd{f_1}{x_n}
    \end{bmatrix} \equiv \nabla f
    = \ma J[1,:] \in\mathbb R^{1\times n}
\end{equation}
we need the first row $\ma J[1,:]$ only. This is also the definition of the
\emph{gradient} $\nabla f(\ve x)\in\mathbb R^n,\: f: \mathbb R^n\ra\mathbb R$.

Note that in the ML literature all partial derivatives including those in
Jacobians and Hessians are often called "gradients", which might confuse you if
you have a physics background :)

\subsection{Vectorized functions}

Vectorized functions (e.g. \verb|sin()| in \numpy, \jax, \pytorch) are a
bit of a special case since, depending on input, they are either $f(x):\mathbb
R\ra\mathbb R$ when given a scalar or $\ve f(\ve x):\mathbb R^n\ra\mathbb R^n$
when given a vector, but never $f(\ve x): \mathbb R^n \ra \mathbb R$ (i.e. the
gradient $\nabla f(\ve x)$ is not defined for them). Importantly we have
$f_i\equiv f$: each $f_i$ is a function of only the $x_i$ it is applied to.
Therefore, the Jacobian is always diagonal, which is a very neat property that
we'll use later.

\subsection{Forward}

Again we have
\begin{equation}
    \ve f(\ve x) = \ve c(\ve b(\ve a(\ve x)))
\end{equation}
and can use the chain rule
\begin{equation}
    \ma J=\pd{\ve f}{\ve x} = \pd{\ve c}{\ve b}\,\pd{\ve b}{\ve a}\,\pd{\ve a}{\ve x}
\end{equation}
which implies a series of matrix multiplications of individual Jacobians $\pdi{\ve
a}{\ve x}$, $\pdi{\ve b}{\ve a}$, $\pdi{\ve c}{\ve b}$.
We can now right-multiply $\ma J$ with the identity
matrix $\pdi{\ve x}{\ve x}$
\begin{equation}
    \pd{\ve f}{\ve x}\,\pd{\ve x}{\ve x} =
    \pd{\ve c}{\ve b}\,\pd{\ve b}{\ve a}\,\pd{\ve a}{\ve x}\,\pd{\ve x}{\ve x}
\end{equation}
which doesn't change anything. We can also think of this as extracting the
$j$-th column $\ma J[:,j]$ by multiplying from the right with one column of the
identity $\partial\ve x/\partial x_j = \ve e_j = \tp{[0,0,\cdots,1,\cdots, 0]}$.
This will thus select the $x_j$ w.r.t. to which we want to calculate the
derivative. Analytically this means for an example $2\times 2$ system and choosing
$x_j = x_1$
\begin{equation}
    \pd{\ve f}{\ve x}\,\pd{\ve x}{x_1}=
    \begin{bmatrix}
        \pd{f_1}{x_1} & \pd{f_1}{x_2} \\
        \pd{f_2}{x_1} & \pd{f_2}{x_2} \\
    \end{bmatrix}
    \begin{bmatrix}
        \pd{x_1}{x_1} \\
        \pd{x_2}{x_1} \\
    \end{bmatrix}
    =
    \begin{bmatrix}
        \pd{f_1}{x_1}\,\pd{x_1}{x_1} + \cancelto{0}{\pd{f_1}{x_2}\,\pd{x_2}{x_1}} \\
        \pd{f_2}{x_1}\,\pd{x_1}{x_1} + \cancelto{0}{\pd{f_2}{x_2}\,\pd{x_2}{x_1}} \\
    \end{bmatrix}
    =
    \begin{bmatrix}
        \pd{f_1}{x_1} \\
        \pd{f_2}{x_1} \\
    \end{bmatrix}
\end{equation}
%
if we note that $\pdi{x_2}{x_1}=0$ and that formally by the chain rule
%
\begin{gather}
    \pd{f_1}{x_1}\,\pd{x_1}{x_1} = \pd{f_1}{x_1} \\
    \pd{f_2}{x_1}\,\pd{x_1}{x_1} = \pd{f_2}{x_1}
    \eqdot
\end{gather}
%
Now, of course we don't want to build up Jacobians. The trick is to evaluate
the expression by using pre-defined derivatives of \emph{primitive}
functions (here $\ve a(\cdot)$, $\ve b(\cdot)$, $\ve c(\cdot)$)\footnote{Any
basic (\numpy) function like \co{sin}, \co{sum} or \co{sqrt} that we can't or
won't represent in terms of even more basic functions.} and by repeated
application of a \emph{\red{Jacobian} \blue{vector} product (JVP)}. We
initialize with $\ve e_j$ and apply JVPs as we go.
%
\begin{splitequation}
    \pd{\ve f}{\ve x}\,\blue{\pd{\ve x}{x_j}} = \pd{\ve f}{\ve x}\,\blue{\ve e_j}
            &= \pd{\ve c}{\ve b}\,\pd{\ve b}{\ve a}\,\red{\pd{\ve a}{\ve x}}\,\blue{\ve e_j} \\
            &= \pd{\ve c}{\ve b}\,\red{\pd{\ve b}{\ve a}}\,\blue{\ve v_j} \\
            &= \red{\pd{\ve c}{\ve b}}\,\blue{\ve v_j'} \\
            &= \ma J[:,j]
\end{splitequation}
%
Now what is a JVP? We see that in order to evaluate the chain rule, each
primitive function $\ve z(\cdot) = \ve a(\cdot), \ve b(\cdot), \ve c(\cdot)$
needs to know ($i$) how it would evaluate its own derivative (Jacobian
$\pdi{\ve z}{\ve{\tau}}$, e.g. $\pdi{\ve b}{\ve a}$) w.r.t. its inputs $\ve{\tau}$ and ($ii$)
how to right-multiply that with a given vector $\ve v$. By doing ($ii$), we can
always initialize the process by $\ve e_j$ and then multiply forward (recall
that forward means $x\ra c$, i.e. from inputs to outputs). Now how do we
do $(i)$? The answer is that we don't, in the sense that we don't need to
instantiate Jacobians. Instead, we \emph{augment} each primitive function $\ve
z(\cdot)$ with a JVP such that we map the so-called
\emph{primal}-\emph{tangent} pair $(\ve x, \ve v)$ to the function at $\ve x$
and its JVP with $\ve v$.
%
\begin{equation}
    (\ve x, \ve v) \mapsto \left(\ve z(\ve x), \red{\pd{\ve z}{\ve\tau}}\,\blue{\ve v}\right)
\end{equation}
%
Each primitive function has an associated JVP function defined (\jax: by using
\co{jax.defjvp}\footnote{The name of this and details of its usage may vary
across \jax versions.}) which gets called when evaluating the chain rule. At
each step through the chain rule along the comp graph, we evaluate the
function's value $\ve z(\ve x)$ \emph{and} its JVP $\red{\pdi{\ve
z}{\ve{\tau}}}\,\blue{\ve v}$ together in lock step. Now, the most
important trick is that the intermediate Jacobians are never explicitly build.
The JVP function only has to return the \emph{result} of $\pdi{\ve
z}{\ve{\tau}}\,\ve v$. Especially, for \numpy's vectorized functions, all
Jacobians are diagonal, which makes implementing $\pdi{\ve z}{\ve{\tau}}\,\ve
v$ simple and efficient. In most cases, we get away with element-wise
operations on \ve v.

Side note: in the AD literature, applying JVPs is also called
"push-forward" of the vector (here $\ve e_j$) from inputs to outputs ($\ve x\ra\ve c$).

When we repeat the process and apply all possible $\ve e_j$ (i.e. the whole
identity matrix), we can build up the full $\ma J$ if we need to, one column at
a time. For the edge case $\ve f(x): \mathbb R^1 \ra \mathbb R^m$ ($n=1$, first column)
we are finished in one forward pass. In general forward mode is efficient for
"tall" Jacobians where $m\gg n$ and inefficient for "wide" ones where $m\ll n$.
The other edge case $f(\ve x): \mathbb R^n \ra \mathbb R^1$ ($m=1$, first row a.k.a.
$\nabla f(\ve x)$) is inefficient in forward mode since we need $n$ passes to
calculate each $\pdi{f}{x_j}$. This has thus similar complexity as finite
differences, where we also repeat $n$ times something like $(f(\ve x +\ve e_j\,h) - f(\ve
x-\ve e_j h))/2\,h$.

\subsection{Reverse}

First, we do a forward pass to trace the execution done in $\ve f$ to build up
the graph $\ve x\ra \ve a\ra \ve b\ra \ve c = \ve f$. Then we can extract the
$i$-th row $\ma J[i,:]$ by multiplying from the left with $\pdi{\ve f}{f_i} =
\ve e_i$. Analytically, we have
\begin{splitequation}
    \left(\pd{\ve f}{f_1}\right)^\top\,\pd{\ve f}{\ve x}
    &=
    \begin{bmatrix}
        \pd{f_1}{f_1} & \pd{f_2}{f_1} \\
    \end{bmatrix}
    \begin{bmatrix}
        \pd{f_1}{x_1} & \pd{f_1}{x_2} \\
        \pd{f_2}{x_1} & \pd{f_2}{x_2} \\
    \end{bmatrix}
    \\
    &=
    \begin{bmatrix}
        \pd{f_1}{f_1}\,\pd{f_1}{x_1} + \cancelto{0}{\pd{f_2}{f_1}\,\pd{f_2}{x_1}} &
        \pd{f_1}{f_1}\,\pd{f_1}{x_2} + \cancelto{0}{\pd{f_2}{f_1}\,\pd{f_2}{x_2}}
    \end{bmatrix}
    \\
    &=
    \begin{bmatrix}
        \pd{f_1}{x_1} & \pd{f_1}{x_2}
    \end{bmatrix}
\end{splitequation}
with $\pdi{f_2}{f_1}=0$.
In the code, we initialize with $\ve e_i$ and this time apply
\emph{\blue{vector} \red{Jacobian} products (VJPs)} as we go.
\begin{splitequation}
    \blue{\left(\pd{\ve f}{f_i}\right)^\top}\,\pd{\ve f}{\ve x} = \blue{\ve e_i^\top}\,\pd{\ve f}{\ve x}
        &= \blue{\ve e_i^\top}\,\red{\pd{\ve c}{\ve b}}\,\pd{\ve b}{\ve a}\,\pd{\ve a}{\ve x}\\
        &= \blue{\ve v_i^\top}\,\red{\pd{\ve b}{\ve a}}\,\pd{\ve a}{\ve x}\\
        &= \blue{\ve v_i'^\top}\,\red{\pd{\ve a}{\ve x}}\\
        &= \ma J[i,:]
\end{splitequation}
Again, we can apply all possible $\ve e_i$ to build up the full $\ma J$, this
time one row at a time. For the edge case $f(\ve x): \mathbb R^n \ra \mathbb
R^1$ ($m=1$, first row) we are done in one backward pass.

Side note: in the AD literature, applying VJPs is also called "pull-back" of
the vector (here $\ve e_i$) from outputs to inputs ($\ve x\la\ve c$).

\subsection{Reverse mode in neural network training a.k.a. backprop}

NN training loss function: a multivariate function $f(\ve x): \mathbb R^n\ra \mathbb R$
\begin{equation}
    f(\ve x) = c(\ve b(\ve a(\ve x)))
\end{equation}
for instance
\begin{minted}{python}
    f = lambda x: c(b(a(x)))
    f = lambda x: sum(power(sin(x),2))
\end{minted}
i.e. the composition of a number of vector-valued functions\footnote{ The NN
training loss' computation graph is of course not short and linear as in our
example.} $\ve a(\cdot), \ve b(\cdot)$ and a final reduction $c(\cdot): \mathbb
R^n\ra \mathbb R$ and where $n$ is "large" (as in $10^6$) and $\ve x$ are the
network's parameters. This is exactly the second edge case from above. When
using reverse mode, we get the gradient $\pdi{f}{\ve x} = \ma J[1,:]$ in a
single reverse pass (i.e. backprop).

\section{Comparison of implementations}

\subsection{\jax, \autograd}

Both are designed to be functional, like \numpy itself. \co{grad()} returns a
function object that evaluates $\nabla f(\ve x)$.
%
\ipmpy{../talk/code/jax_ad_teaser_grad_usage.py}
%
\jax returns \co{DeviceArray} objects because it uses \tf's XLA compiler.
\autograd returns plain \numpy arrays.

\subsection{\pytorch}
%
\pytorch is not functional\footnote{However, there is the experimental
\co{torch.autograd.functional} (checked: v1.8.1).}, it thinks in terms of
\co{torch.Tensor} objects, so when using our little example from above, now
written in \pytorch syntax
%
\ipmpy{../talk/code/pytorch_fwd_rev_1.py}
%
we use the "loss tensor" \co{c} of shape \co{(1,)} instead of the loss function
\co{f}. With knowledge of reverse mode, we can make sense of \pytorch's API
design. Below, we unpack the calculation step by step. Observe how each tensor
has a \verb|grad_fn| attribute which corresponds to defined VJPs. In \pytorch
the tracing of forward ops is implemented such tensors record
operations applied to them.
%
\ipmpy{../talk/code/pytorch_fwd_rev_2.py}
%
Then, we call the \co{backward()} method of the final output tensor, here
\co{c}.
%
\ipmpy{../talk/code/pytorch_fwd_rev_3.py}
%
This "pulls back" the derivatives (starting with $\pdi{f}{f}=1$) from the
output \co{c} to the input and the derivatives are stored in $\pdi{c}{\ve x} =
\co{x.grad}$.\footnote{For a simple linear graph that feels super weird,
pull-back or not. \co{c.grad} would be much more intuitive. For a multi-leaf
DAG, where leafs = input tensors, it makes a bit more sense.}

We can also perform VJPs by calling \co{backward()} of an intermediate
vector-valued variable by giving it a vector argument ($\ve
e$ or $\ve v$ above). In particular, we can build up the intermediate
Jacobians. Here, we extract the first row of $\pdi{\ve b}{\ve a}$.
%
\ipmpy{../talk/code/pytorch_rev_detail_2.py}
%
And finally, the default for "seeding" the backward pass starting at \co{c}
is indeed \co{1.0} as in $\pdi{f}{f}=1$.
%
\ipmpy{../talk/code/pytorch_rev_detail_1.py}

\subsection{Implementations of JVPs (forward) and VJPs (reverse)}

\subsubsection{\autograd}

They define a VJP for "each" numpy function in
\url{https://github.com/HIPS/autograd/blob/master/autograd/numpy/numpy_vjps.py}
using \co{defvjp}, also some JVPs in
\url{https://github.com/HIPS/autograd/blob/master/autograd/numpy/numpy_jvps.py}
but \autograd is mostly only reverse mode. \numpy API definitions in
\url{https://github.com/HIPS/autograd/blob/master/autograd/numpy/numpy_wrapper.py}

\subsubsection{\jax}

\numpy primitives are defined in
\url{https://github.com/google/jax/blob/master/jax/_src/lax/lax.py} with a lot
of \co{defjvp}, one for each primitive, i.e. only forward mode.

\ipmpy{../talk/code/jax_lax_sin.py}

They don't define explicit VJPs (for reverse mode), instead they use the
forward trace, which has to be done anyway, and then run that backwards,
transposing the JVP operations (note that $\tp{\ve e_i}\,\ma J =
\tp{\left(\tp{\ma J}\,\ve e_i\right)}$). From the docs: "\emph{... when
computing reverse differentiation JAX obtains a trace of primitives that
compute the tangent using forward differentiation. Then, JAX interprets this
trace abstractly backwards and for each primitive it applies a transposition
rule.}".

The \numpy API def is in
\url{https://github.com/google/jax/blob/master/jax/_src/numpy}.

\subsubsection{\pytorch}

\pytorch's AD system is designed with focus on the use case $f: \mathbb
R^n\ra\mathbb R$, i.e. the loss function during NN training, where $n$ is
large, which is why is uses VJPs (as in \autograd). VJP definitions are not
done at the Python level as in \jax, using a \co{defvjp} machinery, but instead
in a \co{yaml} file
\url{https://github.com/pytorch/pytorch/blob/master/tools/autograd/derivatives.yaml}
with Python-ish pseudo code, which defines the VJP for each \co{torch}
primitive function. Then scripts are used to generate C++ code for each
function's VJP and Python bindings (e.g. in \co{libtorch.so}).

\subsection{Tracing and compilation}

\subsubsection{\jax}

\jax uses an intermediate representation dubbed \co{jaxpr} which is the result
of tracing a function. Also every other function, e.g. \co{grad(f)} is
represented in \co{jaxpr}.

\begin{minted}{python}
    >>> import jax; from jax import numpy as jnp
    >>> f=lambda x: jnp.sum(jnp.power(jnp.sin(x),2.0))
    >>> jax.make_jaxpr(f)(1.0)
    { lambda  ; a.
      let b = sin a
          c = pow b 2.0
          d = reduce_sum[ axes=() ] c
      in (d,) }

    >>> jax.make_jaxpr(jax.grad(f))(1.0)
    { lambda  ; a.
      let b = sin a
          c = pow b 1.0
          d = mul 2.0 c
          e = mul 1.0 d
          f = cos a
          g = mul e f
      in (g,) }
\end{minted}

It uses \tf's XLA compiler, which uses LLVM, to compile for CPU and GPU targets.

\subsubsection{\pytorch}

\pytorch's \co{jit} creates TorchScript, which is a serialized version of
the graph. That can be exported to disk, loaded into C++ using the libtorch C++
API and executed as a C++ program.

\pytorch can also run on XLA devices like TPUs using
\url{https://github.com/pytorch/xla}.

\section{No Free Lunch: Limitations of AD}
\begin{itemize}
    \item \tf \co{1.x}: Manually construct computation graph by framework
        mini-language (e.g. \co{tf.while}, \co{tf.cond}). \pytorch, \jax,
        \tf\co{2.x}: eager execution / build graph dynamically
    \item In-place ops such as \co{A *= 3} instead of
        \co{B = A*3} are hard to support and can slow things down because large
        parts of the comp graph must be rewritten instead of just adding a new
        variable. In \autograd, \jax and \pytorch, usage of in-place mods is
        discouraged and will often just raise an exception. \pytorch has an
        elaborate tensor versioning system that will cause backward grad
        calculations to fail when in-pace ops would break it
        (\url{https://pytorch.org/docs/stable/notes/autograd.html#in-place-operations-with-autograd}).
        \jax and \pytorch have a \verb|jit|, which could optimize out-of-place
        modifications and copies away if possible, not sure if they do.
    \item Watch out when ADing through fast but approximate
        implementations: "Approximate the derivative, not differentiate the
        approximation." (see \verb|examples/text_jax.py|)
    \item For end-to-end AD, all parts of a pipeline must be implemented using
        the same AD-aware library. Else one needs to cut the chain rule open
        and hand over derivatives at the interface between codes.
\end{itemize}

\newpage
\nocite{*}
\printbibliography

\end{document}
