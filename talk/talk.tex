% vim:tw=150

\pdfminorversion=5
\documentclass[fleqn,10pt]{beamer}

%----------------------------------------------------------------------------------%

\usepackage{bm}
\usepackage{xspace}
\usepackage{amsmath}
%%\usepackage{footmisc}
\usepackage{minted}
\usepackage{cancel}

\usepackage[
    backend=biber,
    maxbibnames=2,
    giveninits=true,
    url=true,
    isbn=false,
    sorting=none,
    date=year,
    ]{biblatex}
\addbibresource{lit.bib}



\newcommand{\ve}[1]{\ensuremath{\bm{\mathit{#1}}}}
\newcommand{\ma}[1]{\ensuremath{\bm{\mathbf{#1}}}}
\newcommand{\dd}{\text{d}}
\newcommand{\ra}{\ensuremath{\rightarrow}\xspace}
\newcommand{\dra}{\ensuremath{\Rightarrow}\xspace}
\newcommand{\la}{\ensuremath{\leftarrow}\xspace}
\newcommand{\pd}[2]{\dfrac{\partial #1}{\partial #2}}
\newcommand{\td}[2]{\dfrac{\dd #1}{\dd #2}}
\newcommand{\pdi}[2]{\partial #1/\partial #2}
\newcommand{\tdi}[2]{\dd #1/\dd #2}
\newcommand{\red}[1]{{\color{red}{#1}}}
\newcommand{\green}[1]{{\color{green}{#1}}}
\newcommand{\blue}[1]{{\color{blue}{#1}}}
\newcommand{\ts}[1]{\textsuperscript{#1}}
\newcommand{\soft}[1]{\textsl{#1}\xspace}
\newcommand{\numpy}{\soft{numpy}}
\newcommand{\pytorch}{\soft{PyTorch}}
\newcommand{\jax}{\soft{JAX}}
\newcommand{\autograd}{\soft{autograd}}
\newcommand{\scipy}{\soft{scipy}}
\newcommand{\numdifftools}{\soft{numdifftools}}
\newcommand{\tf}{\soft{TensorFlow}}
\newcommand{\sympy}{\soft{sympy}}
\newcommand{\co}[1]{\texttt{#1}}
\newcommand{\norm}[1]{\left\lVert#1\right\rVert}
\newcommand{\eqdot}{\:.}


%----------------------------------------------------------------------------------%

%----------------------------------------------------------------------------------%

\defbeamertemplate{footline}{megamega}{%
    \usebeamercolor[fg]{page number in head/foot}%
    \usebeamerfont{page number in head/foot}%
%%    \hspace{1em}HZDR ML Paper Reading Group \hspace{1em} Steve Schmerler \hspace{1em} FWCC/Helmholtz AI
    \hfill%
    \insertframenumber\,/\,\inserttotalframenumber\kern1em\vskip2pt%
}
\setbeamertemplate{footline}[megamega]{}

\setbeamerfont{footline}{size=\fontsize{6}{6}\selectfont}
\setbeamertemplate{navigation symbols}{}
\setbeamertemplate{caption}{\insertcaption}

\usefonttheme{serif}

%----------------------------------------------------------------------------------%

\title{Autodiff 101}
\subtitle{Introduction to Automatic Differentiation with examples from \jax and \pytorch}

\author{Steve Schmerler}
\institute{FWCC/Helmholtz AI}

\date{}

%----------------------------------------------------------------------------------%

\begin{document}

\begin{frame}
    \titlepage
\end{frame}

\begin{frame}
    \frametitle{Motivation}
    \begin{block}{Why derivatives?}
        In DL: $f(\ve x) = L(\ve\theta),\quad f:\mathbb R^n\ra\mathbb R,\quad \min_{\ve x}f(\ve x)$ using SGD $\ra\nabla_{\ve x} f$
    \end{block}
    \begin{block}{How to get $\nabla f$}
    \begin{itemize}
        \item work out derivatives by hand (or really \soft{Mathematica}, \sympy, ...) and code them
        \item call symbolic engine (\soft{sympy}, ...)
        \item use finite diff approximation $[f(\ve x +\ve e_j\,h) - f(\ve x-\ve e_j h)]/2\,h$
        \item AD: diff through mathematical operations in (arbitrary) source code using
        \begin{itemize}
            \item chain rule
            \item pre-defined derivatives of elementary ("\emph{primitive}") functions
        \end{itemize}
        \item in NNs: "backprop" = reverse mode AD
    \end{itemize}
    \end{block}
    \begin{columns}
        \column{0.5\textwidth}
            \centering
            \includegraphics[width=0.8\textwidth]{pics/pytorch-logo-dark.png}
        \column{0.5\textwidth}
            \centering
            \includegraphics{pics/jax_logo_250px.png}\\
            \url{github.com/google/jax}
    \end{columns}
\end{frame}


\begin{frame}[fragile]
    \frametitle{Chain rule}
    \begin{equation*}
        f(x) = c(b(a(x)))
    \end{equation*}

    \begin{equation*}
        f(x) = \ln(\sin^2(x))
    \end{equation*}

    \begin{minted}{python}
    f = lambda x: np.log(np.power(np.sin(x), 2))
    \end{minted}

    \begin{equation*}
        \td{f}{x} = \red{\td{c}{b}}\,\green{\td{b}{a}}\,\blue{\td{a}{x}}
            = \red{\frac{1}{\sin^2(x)}}\,\green{2\,\sin(x)}\,\blue{\cos(x)}
            = 2\,\frac{\cos(x)}{\sin(x)} = 2\,\cot(x)
    \end{equation*}

    \begin{minted}{python}
    fprime = lambda x: 2 * np.cos(x) / np.sin(x)
    \end{minted}
\end{frame}


\begin{frame}[fragile]
    \frametitle{AD teaser: arbitrary code}
    $f:\mathbb R^n\ra\mathbb R$
    \begin{minted}{python}
    def f(x):
        if np.sum(x) > 1.234:
            tmp = np.log(np.power(np.sin(x), 2))
        else:
            tmp = -10 * np.tan(x)
        return np.sum(np.exp(tmp) * 2*np.pi)
    \end{minted}
    \vfill
\end{frame}


\begin{frame}[fragile]
    \frametitle{AD teaser: arbitrary code}
    \begin{minted}{python}
    from jax import grad
    import jax.numpy as np

    def f(x):
        if np.sum(x) > 1.234:
            tmp = np.log(np.power(np.sin(x), 2))
        else:
            tmp = -10 * np.tan(x)
        return np.sum(np.exp(tmp) * 2*np.pi)
    \end{minted}
    \vfill
    \begin{minted}{python}
    >>> g = grad(f)

    # f: R -> R
    >>> g(23)
    DeviceArray(5.666104, dtype=float32)

    # f: R^n -> R
    >>> g(rand(3))
    DeviceArray([6.025256, 5.720606, 6.281912], dtype=float32)
    \end{minted}
\end{frame}

\begin{frame}
    \frametitle{Chain rule for scalar $f$, AD terminology}
    \begin{equation*}
        f(x) = c(b(a(x)))
    \end{equation*}
    Computation graph: $x\ra a\ra b\ra c = f$
    \begin{align*}
        \td{f}{x}
            &= \td{c}{b}\,\td{b}{a}\,\td{a}{x}&\\
            &= \td{c}{b}\,\left(\td{b}{a}\,\left(\td{a}{x}\right)\right)& \text{forward: inputs $x$ \ra outputs $c$}\\
            &= \left(\left(\td{c}{b}\right)\,\td{b}{a}\right)\,\td{a}{x}& \text{reverse: inputs $x$ \la outputs $c$}\\
    \end{align*}
\end{frame}


\begin{frame}[fragile]
    \frametitle{Jacobian}
    \begin{equation*}
        \ve f: \mathbb R^n \ra \mathbb R^m, \ve x\mapsto \ve f(\ve x) = \left[f_1(\ve x), \cdots, f_m(\ve x)\right]
    \end{equation*}
    \begin{equation*}
        \ve f(\ve x) = \ve c(\ve b(\ve a(\ve x)))
    \end{equation*}
    For $n=m$:
    \begin{minted}{python}
    f = lambda x: np.log(np.power(np.sin(x), 2))
    \end{minted}

    \begin{equation*}
        \ve f: \mathbb R^n \ra \mathbb R^m,\quad \ma J
        = \pd{\ve f}{\ve x} =
        \begin{bmatrix}
            \pd{f_1}{x_1} & \cdots & \pd{f_1}{x_n}  \\
            \vdots        & \ddots & \vdots         \\
            \pd{f_m}{x_1} & \cdots & \pd{f_m}{x_n}
        \end{bmatrix}
        \in\mathbb R^{m\times n}
    \end{equation*}
    Note: $\pd{\ve x}{\ve x} = (\delta_{ij})$ is the identity matrix
\end{frame}

\newcommand{\edgecol}{
    \begin{equation*}
        \ve f: \mathbb R \ra \mathbb R^m\quad \ma J
        = \pd{\ve f}{x} =
        \begin{bmatrix}
            \pd{f_1}{x_1}  \\
            \vdots         \\
            \pd{f_m}{x_1}
        \end{bmatrix}
        = \ma J[:,1]\in\mathbb R^{m\times 1}
    \end{equation*}
}

\newcommand{\edgerow}{
    \begin{equation*}
        f: \mathbb R^n \ra \mathbb R\quad \ma J
        = \pd{f}{\ve x} =
        \begin{bmatrix}
            \pd{f_1}{x_1} & \cdots & \pd{f_1}{x_n}
        \end{bmatrix} \equiv \nabla f
        = \ma J[1,:] \in\mathbb R^{1\times n}
    \end{equation*}
}

\begin{frame}
    \frametitle{Jacobian edge cases}
    $n=1$
    \edgecol
    \uncover<2>{
        $m=1$
        \edgerow
    }
\end{frame}


\begin{frame}
    \frametitle{Forward mode AD}
    \begin{equation*}
        \ve f: \mathbb R^n \ra \mathbb R^m,\quad\ve f(\ve x) = \ve c(\ve b(\ve a(\ve x)))
    \end{equation*}
    \begin{equation*}
        \ma J=\pd{\ve f}{\ve x} = \pd{\ve c}{\ve b}\,\pd{\ve b}{\ve a}\,\pd{\ve a}{\ve x}
    \end{equation*}
    \uncover<2->{
        Right-multiply with identity matrix $\pdi{\ve x}{\ve x}$
        \begin{equation*}
            \pd{\ve f}{\ve x}\,\pd{\ve x}{\ve x} =
            \pd{\ve c}{\ve b}\,\pd{\ve b}{\ve a}\,\pd{\ve a}{\ve x}\,\pd{\ve x}{\ve x}
        \end{equation*}
    }
    \uncover<3->{
        Right-multiply with one column $\pdi{\ve x}{x_j} = \ve e_j = [0,0,\cdots,1,\cdots, 0]$ selects $j$th column $\ma J[:,j]$, e.g. $x_j=x_1$
    }
    \uncover<4->{
        {\small
        \begin{equation*}
            \pd{\ve f}{\ve x}\,\pd{\ve x}{x_1}=
            \begin{bmatrix}
                \pd{f_1}{x_1} & \pd{f_1}{x_2} \\
                \pd{f_2}{x_1} & \pd{f_2}{x_2} \\
            \end{bmatrix}
            \begin{bmatrix}
                \pd{x_1}{x_1} \\
                \pd{x_2}{x_1} \\
            \end{bmatrix}
            =
            \begin{bmatrix}
                \pd{f_1}{x_1}\,\pd{x_1}{x_1} + \cancelto{0}{\pd{f_1}{x_2}\,\pd{x_2}{x_1}} \\
                \pd{f_2}{x_1}\,\pd{x_1}{x_1} + \cancelto{0}{\pd{f_2}{x_2}\,\pd{x_2}{x_1}} \\
            \end{bmatrix}
            =
            \begin{bmatrix}
                \pd{f_1}{x_1} \\
                \pd{f_2}{x_1} \\
            \end{bmatrix}
        \end{equation*}
        }
    }
\end{frame}

\begin{frame}
    \frametitle{Forward mode AD: Jacobian-vector products}
    \begin{align*}
        \pd{\ve f}{\ve x}\,\pd{\ve x}{x_j} = \pd{\ve f}{\ve x}\,\ve e_j
            &= \pd{\ve c}{\ve b}\,\pd{\ve b}{\ve a}\,\blue{\pd{\ve a}{\ve x}\,\ve e_j} \\
            &= \pd{\ve c}{\ve b}\,\blue{\pd{\ve b}{\ve a}\,\ve v_j} \\
            &= \blue{\pd{\ve c}{\ve b}\,\ve v_j'} \\
            &= \ma J[:,j]
    \end{align*}
    \uncover<2->{
        Augment each intermediate $\ve z(\cdot) = \ve a(\cdot), \ve b(\cdot), \ve c(\cdot)$ with
        \begin{equation*}
            \co{jvp}: (\ve x, \ve v) \mapsto \left(\ve z(\ve x), \ma J_{\ve z}(\ve x)\,\ve v\right)
        \end{equation*}
    }
    \uncover<3->{
        {\small
        \begin{itemize}
            \item $\ve z(\ve x)$ and JVP evaluated together along graph $x\ra a\ra b\ra c$
            \item define JVP for each primitive \numpy func $\ve z(\cdot)$
            \item $\ma J_{\ve z}$ is \emph{never} build, diagonal for most \numpy funcs, JVP are simple expressions (e.g. element-wise ops on $\ve v$)
        \end{itemize}}
    }
\end{frame}

\begin{frame}[fragile]
    \begin{minted}{python}
import jax
import jax.numpy as np

@jax.custom_jvp
def mysin(x):
    return _my_awesome_sin(x)

@mysin.defjvp
def mysin_jvp(primals, tangents):
    x, = primals
    v, = tangents
    return mysin(x), np.cos(x) * v

@mysin.defjvp
def mysin_jvp_with_jac(primals, tangents):
    x, = primals
    v, = tangents
    ##jac = jax.jacobian(np.sin)(x)
    jac = np.diag(np.cos(x))
    return mysin(x), np.dot(jac, v)
    \end{minted}
\end{frame}

\begin{frame}
    \frametitle{Properties of forward mode AD}
    Apply all $\ve e_j$ \ra build $\ma J$ \emph{one column at a time}.
    Efficient for "tall" $\ma J$ with $m\gg n$, best: single forward pass for $\ve f(x): \mathbb R \ra \mathbb R^m$ ($n=1$, first column)
    \edgecol
    Inefficient for "wide" $\ma J$ with $m\ll n$, worst: $n$ forward passes for $f(\ve x): \mathbb R^n \ra \mathbb R$ ($m=1$, first row)
    \edgerow
    Same as FD $[f(\ve x +\ve e_j\,h) - f(\ve x-\ve e_j h)]/2\,h$.
\end{frame}

\begin{frame}
    \frametitle{Reverse mode AD: vector-Jacobian products}
    Multiplying from the left with $\pdi{\ve f}{f_i} = \ve e_i$
    \begin{align*}
        \left(\pd{\ve f}{f_i}\right)^\top\,\pd{\ve f}{\ve x} = \ve e_i^\top\,\pd{\ve f}{\ve x}
            &= \blue{\ve e_i^\top\,\pd{\ve c}{\ve b}}\,\pd{\ve b}{\ve a}\,\pd{\ve a}{\ve x}\\
            &= \blue{\ve v_i^\top\,\pd{\ve b}{\ve a}}\,\pd{\ve a}{\ve x}\\
            &= \blue{\ve v_i'^\top\,\pd{\ve a}{\ve x}}\\
            &= \ma J[i,:]
    \end{align*}
    \uncover<2->{
        Augment each intermediate $\ve z(\cdot) = \ve a(\cdot), \ve b(\cdot), \ve c(\cdot)$ with
        \begin{equation*}
            \co{vjp}: (\ve x, \ve v) \mapsto \ve v\,\ma J_{\ve z}(\ve x)
        \end{equation*}
    }
    \uncover<3->{
        {\small
        \begin{itemize}
            \item 2 phases
            \begin{itemize}
                \item (1) forward pass to trace graph $x\ra a\ra b\ra c$ ("taping")
                \item (2) backward pass: apply VJPs
            \end{itemize}
            \item define VJPs for each primitive \numpy func $\ve z(\cdot)$
        \end{itemize}}
%%        Note that
%%        \begin{equation*}
%%            \ve e_i^\top\,\ma J = \ma J^\top\,\ve e_i
%%        \end{equation*}
    }
\end{frame}

\begin{frame}[fragile]
    \frametitle{Properties of reverse mode AD}
    Apply all $\ve e_i$ \ra build $\ma J$ \emph{one row at a time}.
    Efficient for "wide" $\ma J$ with $m\ll n$, best: one single reverse pass for $f(\ve x): \mathbb R^n \ra \mathbb R$ ($m=1$, first row)
    \edgerow
    DL: loss function $f(\ve x): \mathbb R^n\ra \mathbb R$
    \begin{equation*}
        f(\ve x) = c(\ve b(\ve a(\ve x)))
    \end{equation*}
    \begin{minted}{python}
    f = lambda x: c(b(a(x)))
    f = lambda x: np.sum(np.power(np.sin(x),2))
    \end{minted}
\end{frame}

\begin{frame}[fragile]
    \frametitle{Reverse mode VJPs in \pytorch}
    \url{https://github.com/pytorch/pytorch/blob/master/tools/autograd/derivatives.yaml}
    \begin{minted}{yaml}
    - name: sigmoid(Tensor self) -> Tensor
      self: sigmoid_backward(grad, result)

    - name: sign(Tensor self) -> Tensor
      self: zeros_like(grad, at::MemoryFormat::Preserve)

    - name: sin(Tensor self) -> Tensor
      self: grad * self.cos()

    - name: sinh(Tensor self) -> Tensor
      self: grad * self.cosh()
    \end{minted}
    Scripts to generate C++ code (\co{libtorch.so}) and Python bindings.
\end{frame}

\begin{frame}[fragile]
    \frametitle{\pytorch: Reverse mode example}
    \begin{equation*}
        f: \mathbb R^n\ra \mathbb R, \quad f(\ve x) = c(\ve b(\ve a(\ve x)))
    \end{equation*}
    \begin{minted}{python}
    c = torch.sum(torch.pow(torch.sin(x),2))
    c = x.sin().pow(2.0).sum()
    \end{minted}
    Forward pass, tracing, observe \verb|grad_fn|
    \begin{minted}{python}
    >>> x = torch.rand(3, requires_grad=True)
    >>> a = torch.sin(x)
    tensor([0.7826, 0.2057, 0.5249], grad_fn=<SinBackward>)
    >>> b = torch.pow(a, 2.0)
    tensor([0.6125, 0.0423, 0.2755], grad_fn=<PowBackward0>)
    >>> c = torch.sum(b)
    tensor(0.9303, grad_fn=<SumBackward0>)
    \end{minted}
    Backward pass
    \begin{minted}{python}
    # same as torch.autograd.grad(c,x)
    >>> c.backward()
    >>> x.grad
    tensor([0.9743, 0.4026, 0.8935])
    \end{minted}
\end{frame}

\begin{frame}[fragile]
    \frametitle{\pytorch: Reverse pass details}
    \begin{equation*}
        f: \mathbb R^n\ra \mathbb R, \quad f(\ve x) = c(\ve b(\ve a(\ve x)))
    \end{equation*}
    \begin{align*}
        \pd{f}{f}\,\pd{f}{\ve x} = 1\,\pd{f}{\ve x}
            &= \blue{1\,\pd{c}{\ve b}}\,\pd{\ve b}{\ve a}\,\pd{\ve a}{\ve x}\\
            &= \blue{\ve v_i^\top\,\pd{\ve b}{\ve a}}\,\pd{\ve a}{\ve x}\\
            &= \blue{\ve v_i'^\top\,\pd{\ve a}{\ve x}}
    \end{align*}
    Default: initialize backward pass with $\pdi{f}{f}=1$
    \begin{minted}{python}
    >>> c.backward(torch.tensor(1.0))
    \end{minted}
    VJP: extract one row of Jacobian $\pd{\ve b}{\ve a}$
    \begin{minted}{python}
    >>> b = torch.pow(a, 2.0)
    >>> v = torch.tensor([1.0, 0.0, 0.0])
    >>> b.backward(v)
    >>> x.grad
    tensor([0.9619, 0.0000, 0.0000])
    \end{minted}
\end{frame}

\begin{frame}[fragile]
    \frametitle{No free lunch}
    \begin{itemize}
        \item in-place ops (\co{a[i] *= 3}) can break AD, \pytorch: Tensor versioning system
        \begin{minted}{python}
    RuntimeError: one of the variables needed for gradient
    computation has been modified by an inplace operation:
    [torch.FloatTensor []], which is output 0 of
    SelectBackward, is at version 55; expected version 51
    instead. Hint: enable anomaly detection to find the
    operation that failed to compute its gradient, with
    torch.autograd.set_detect_anomaly(True).
        \end{minted}
        \item numerical accuracy a.k.a. when do you want \verb|custom_jvp|
        \begin{itemize}
            \item AD derivatives can generate numerically unstable code
            \item example: ADing fast but inaccurate approximations of functions (e.g. \co{exp(x)}): "approximate the derivative, not differentiate
                the approximation"
        \end{itemize}
    \end{itemize}
\end{frame}


\begin{frame}
    \nocite{*}
    \printbibliography
\end{frame}


\end{document}
